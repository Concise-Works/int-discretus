\section{Prerequisites}
\begin{note}
    It's about \textbf{understanding}, \textbf{NOT remembering}. Read definitions, know they exist, and as you work through problems, revisit them. You'll remember more that way.
\end{note}
Definitions are everything. Misunderstanding just a word may send you on a wild goose chase.\\

\begin{definition}
    ``$a := b$" \ a contraction of the phrase ``$a$ is to be defined as $b$."  
\end{definition}

\begin{definition}
    For integers $a$ and $b$, $a \mid b:=$ ``$a$ \textbf{divides} $b$",  
    meaning $b = ak$ for some integer $k$.
    \textbf{e.g.,} $2 \mid 12$, as $2$ times \textit{some} number equals 12: $12 = 2(6)$.
\end{definition}

\begin{definition} 
    ``\textbf{Multiple.}" \ For integers $a$ and $b$, $b$ is a multiple of $a$ if $a \mid b$.\\
    \textbf{e.g.,} 12 is a multiple of 6 because $6 \mid 12$.\\
    \textbf{i.e.,} $b$ is a multiple if it can be produced by $a$ times some number.
\end{definition}

\begin{definition}
    ``\textbf{Common Multiple.}" \ For integers $a$, $b$, and $c$;\\
    $c$ is a common multiple of $a$ and $b$, if $a \mid c$ and $b \mid c$.\\
    \textbf{e.g.,} 12 is a common multiple of 3 and 4.
\end{definition}

\begin{definition}
    \textbf{Multiplication Terms.} \\ in $5\times6=30$: the
    \textbf{Multiplicand}: 5,
    \textbf{Multiplier}: 6,
    \textbf{Product}: 30.\\
    Order of Multiplicand and Multiplier don't matter as we'll show later.
\end{definition}

\begin{definition}
    \textbf{Division Terms.} Let $Rx:=$ ``for some integer x, x is the remainder"\\
    in ``$20\div6=3\ R2$" , \textbf{i.e.}, $\frac{23}{6}=3\ R2$:\\ the
    \textbf{Dividend}: 20, 
    \textbf{Divisor}: 6, 
    \textbf{Quotient}: 3,
    \textbf{Remainder}: 2,\\
    i.e., Divisor $\mid$ (Dividend + Remainder), as $3\mid(20+3)$.
\end{definition}


\newpage


\begin{definition}
    ``\textbf{Floor.}" \ The \textbf{floor} of a real number $a$, denoted $\lfloor a \rfloor$, is the greatest integer less than or equal to $a$.\\
    \textbf{e.g.,} $\lfloor \mathrm{3.\.14} \rfloor = 3$ and $\lfloor \mathrm{-3.\.14} \rfloor = -4$.
\end{definition}

\begin{definition}
    ``\textbf{Ceiling.}" \ The \textbf{ceiling} of a real number $a$, denoted $\lceil a \rceil$, is the smallest integer greater than or equal to $a$.\\
    \textbf{e.g.,} $\lceil \mathrm{3.\.14} \rceil = 4$ and $\lceil \mathrm{-3.\.14} \rceil = -3$.
\end{definition}

\begin{definition}
    ``\textbf{Composite.}" \ An integer $a$ is a composite if it is divisible by some positive integer other than 1 or itself.
\end{definition}
 