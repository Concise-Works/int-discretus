\vspace{-1cm}
\createcover{Propositional Logic}{8cm}{
    \textbf{Propositional Logic} is the branch of knowledge that concerns itself with taking regular everyday expressions like ``if it rains today I'm going to bring an umbrella," \ into logical statements, i.e., propositions, that can be manipulated and further abstracted to mold into even more complex expressions.
}

\subsection{Truth Tables}
Below is a truth table with variables $x$, $y$ and $z$. The table represents the relationship between these variables, in-turn describing a function. Let's first begin by explaining the symbols.

\begin{definition}
    \begin{itemize}\!
        \item  \textbf{1} represents \textbf{true}, and \textbf{0} \textbf{false}.
        \item  $\lor$ represents ``or."
        \item  $\land$ represents ``and." \ Think `A' for ``And"\! as `$\land$' looks like an `A'.
        \item $\neg$ means negation, i.e., ``not."
    \end{itemize}
\end{definition}

\vspace{-1.5em} % Adjust this value as needed to reduce space

% Table with figure name and label

% Table with figure name and label
\begin{table}[htbp]
    \caption{}
    \label{tab:truth-table}
    \[
    \bgroup
    \def\arraystretch{1.5} % This adds padding
    \newcolumntype{Y}{>{\centering\arraybackslash}X} % This centers the text
    \newcolumntype{Z}{>{\centering\arraybackslash}p{2cm}} % This creates a new column type with less space
    \newcolumntype{W}{>{\centering\arraybackslash}p{6cm}} % Column type for the last column with fixed width
    \begin{tabularx}{\textwidth}{|Z|Z|Z|Y|} % Use the new column type for x, y, z, and the last column
    \hline
    x & y & z & ($\neg x\land \neg y \land z$)$\lor$($x \land \neg y \land z$) \\
    \hline
    0 & 0 & 0 & 0\\
    \hline
    \rowcolor{lightgray} 0 & 0 & 1 & 1\\ 
    \hline
    0 & 1 & 0 & 0\\
    \hline
    0 & 1 & 1 & 0\\
    \hline
    1 & 0 & 0 & 0\\
    \hline
    \rowcolor{lightgray} 1 & 0 & 1 & 1\\
    \hline
    1 & 1 & 0 & 0\\
    \hline
    1 & 1 & 1 & 0\\
    \hline
    \end{tabularx}
    \egroup
    \]
\end{table}

\vspace{-2em} % Adjust this value as needed to reduce space
 \textbf{The last column represents the ``function" or ``expression."} Columns proceeding left break the function into smaller manageable chunks.\\
 \underline{Highlighted rows show when the function is true}, as follows:
 
\begin{strat}
\vspace{-1em} % Adjust this value as needed to reduce space
       \begin{align*}
        (x=0 \text{ and } y=0 \text{ and } z=1) \text{ or when } (x=1 \text{ and } y=0 \text{ and } z=1).& \text{ i.e.,}\\
        (x=0\land y=0\land z=1)\lor(x=1 \land y=0\land z=1).& \text{ Assume inputs to be 1}\\
        (\neg x\land \neg y\land z)\lor(x \land \neg y\land z).& \text{ Negate 0's}
    \end{align*}
\end{strat}
\vspace{-1em} % Adjust this value as needed to reduce space
Which yields the function that the table shows. 

\begin{note}
    Truth tables come in handy for brute forcing a problem you're stuck. Any logical function can be expressed using truth tables. However, this quickly becomes bewildering as the size of the table will quickly grow larger than the number of atoms in the universe.\\
    
    This is known as the NP-completeness problem, a concept discovered by Leonid Levin and Stephen Cook. That's where Computer science opens new doors in developing efficient algorithms that can handle large-scale problems. There's that for some math motivation.
\end{note}


\newpage

\textbf{Behaviors of truth tables}
\vspace{.3em} % Adjust this value as needed to reduce space
\hrule % This creates a horizontal line outside the table

\[
\bgroup
\def\arraystretch{1.2}
\begin{tabularx}{\textwidth}{XXX}
    \textbf{Table 1} & \textbf{Table 2} & \textbf{Table 3}\\
    \begin{tabular}{|c|}
    \hline
    x \\
    \hline
    1 \\
    \hline
    0 \\
    \hline
    \end{tabular}
    &
    \begin{tabular}{|c|c|}
    \hline
    x & y \\
    \hline
    1 & 1 \\
    \hline
    1 & 0 \\
    \hline
    0 & 1 \\
    \hline
    0 & 0 \\
    \hline
    \end{tabular}
    &
    \begin{tabular}{|c|c|c|}
    \hline
    x & y & ... \\
    \hline
    0 & 0 & ... \\
    \hline
    1 & 0 & ... \\
    \hline
    0 & 1 & ... \\
    \hline
    1 & 1 & ... \\
    \hline
    ... & ... & ... \\
    \hline
    \end{tabular}
    \\
\end{tabularx}
\egroup
\]
\hrule % This creates a horizontal line outside the table

\textbf{You may choose to start with a 1 or 0}, though you must stay consistent in that choice. It's about getting all possible combinations between variables. \underline{The rules for which are:}

\begin{definition}
    The \textbf{number of rows} in a truth table grow $\bm{2^n}$, for 
    \underline{$n$ number of variables}.\\
    \textbf{e.g.}, $(n=4)$, there are $2^4 = 16$ rows.
\end{definition}

\begin{definition}
    The number of \textbf{Consecutive 0s or 1s} grows by $2^{n-1}$, for the \underline{$n^{th}$ variable.}\\
    \textbf{e.g.}, $(n=4)$, there are $2^{(4-1)} = 8$ consecutive 0s or 1s.
\end{definition}

 Choose the starting variable, then proceeding variables will be the next $n^{th}$ variable.e.i, in 


\subsection{Conjunctive and Disjunctive Normal Form}
\textbf{CNF} and \textbf{DNF} are abbreviations for \underline{Conjunctive Normal Form and Disjunctive Normal Form}, respectively. These forms describe how logical functions are structured.\\

\begin{note}
    A \textbf{literal} is a basic, indivisible element in propositional logic. It is either an atomic proposition (a variable) or its negation. For example, if $x$ is a proposition, then $x$ and $\neg x$ (not $x$) are both literals.
\end{note}

\begin{definition}
    \textbf{DNF ($\lor$):} represents a function as a disjunction (OR) of one or more conjunctions (AND) of literals. Each term in DNF is a conjunction of literals, and these terms are combined using OR.\\
\end{definition}
\begin{definition}
    \textbf{CNF ($\land$): } represents a function as a conjunction (AND) of one or more disjunctions (OR) of literals. Each term in CNF is a disjunction of literals, and these terms are combined using AND.\\
\end{definition}


\[
\bgroup
\def\arraystretch{1.5}
\newcolumntype{Y}{>{\centering\arraybackslash}X}
\begin{tabularx}{\textwidth}{|Y|Y||Y||Y|Y|}
    \hline
    \multicolumn{2}{|c||}{ CNF }& Both &\multicolumn{2}{c|}{ DNF }\\
    \hline
    True & False & True & False & True\\
    \hline
    $\overline{x}(x+\overline{y})(w)$ & $z(xy+\overline{w})$ & $x+y+z$ & $\overline{x+y}$ & $xy+\overline{w}$ \\
    \hline
    $(x+y+z)\overline{w}$ & $\overline{(x+y)}(x+y)$ & $xy\overline{w}$ & $(x+y)\overline{w}$ & $(xyz)+(\overline{x}y)$\\
    \hline
    $\overline{x}+y$& $\overline{x(y+z)}$  & $\overline{x}$  & $\overline{x+(yz)}$ &$\overline{xy}$\\
    \hline
\end{tabularx}
\egroup
\]
\textbf{Note:} for the example $\overline{x+y}$, the complement operation is applied to the sum $x + y$. In disjunctive and conjunctive normal form, the complement operation can only be applied to single variables.
\\
\[
\bgroup
\def\arraystretch{1.5}
\newcolumntype{Y}{>{\centering\arraybackslash}X}
\begin{tabularx}{\textwidth}{|Y|Y|Y|Y|Y|Y|Y|Y|}
    \hline
    \multicolumn{4}{|c|}{ CNF to DNF }&\multicolumn{4}{c|}{ DNF to CNF }\\
    \multicolumn{4}{|c|}{ $(x+y+z)(x+\overline{z})(\overline{y}+z)$ }&\multicolumn{4}{c|}{ $x\overline{y}z+x\overline{z}+\overline{x}y$ }\\
    \hline
    x & y & z & $f(x)$ & x & y & z & $f(x)$ \\
    \hline
    0 & 0 & 0 & 0 & \rowcolor{lightgray}0 & 0 & 0 & 0 \\
    \hline
    0 & 0 & 1 & 0 & \rowcolor{lightgray}0 & 0 & 1 & 0\\
    \hline
    0 & 1 & 0 & 0 & 0 & 1 & 0 & 1\\
    \hline
    0 & 1 & 1 & 0 & 0 & 1 & 1 & 1\\
    \hline
    \rowcolor{lightgray}1 & 0 & 0 & 1 & \rowcolor{white}1 & 0 & 0 & 1\\
    \hline
    \rowcolor{lightgray}1 & 0 & 1 & 1 & \rowcolor{white}1 & 0 & 1 & 1\\
    \hline
    1 & 1 & 0 & 0 & 1 & 1 & 0 & 1\\
    \hline
    \rowcolor{lightgray}1 & 1 & 1 & 1 & 1 & 1 & 1 & 0 \\
    \hline
    \multicolumn{4}{|c|}{ $x\overline{y}\,\overline{z}+x\overline{y}z+xyz$ }& \multicolumn{4}{c|}{ $(x+y+z)(x+y+\overline{z})(\overline{x}+\overline{y}+\overline{z})$ }\\
    \hline
\end{tabularx}
\egroup
\]
\textbf{Notice:} \textbf{CNF to DNF}, look for \underline{positive outputs} based on the equation above, then transcribe them \underline{flipping 0 inputs to positive}. Then vise-versa with \textbf{DNF to CNF}.


\subsection{Functional Completeness}
\label{subsec:subsection0}
for a function to be functionally complete it can be expressed using only:\\ \{AND (conjunction), OR (disjunction), NOT (negation)\}. \\For example $\uparrow$(NAND) and $\downarrow$(NOR) are functionally complete, because they as functions can be expressed using only the set: 
NAND $\equiv \neg(p\land q)$ and NOR $\equiv \neg(a\lor b)$.\\