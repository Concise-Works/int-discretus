% Tcolorboxes
\makeatother
\usepackage{thmtools}
\usepackage[framemethod=TikZ]{mdframed}
\mdfsetup{skipabove=1em,skipbelow=1em}

\theoremstyle{definition}
\declaretheoremstyle[
    headfont=\bfseries\sffamily\color{black!70!black}, bodyfont=\normalfont,
    mdframed={
        linewidth=2pt,
        rightline=true, topline=false, bottomline=false,
        linecolor=black, backgroundcolor=black!3!white,
    }
]{thmbox}

\declaretheoremstyle[
    headfont=\bfseries\sffamily\color{black!3}, bodyfont=\normalfont,
    mdframed={
        linewidth=2pt,
        rightline=false, topline=false, bottomline=false,
        linecolor=black, backgroundcolor=black!3!white,align=center,
    }
]{thmtikz}


\declaretheoremstyle[
    headfont=\bfseries\sffamily\color{black!70!black}, bodyfont=\normalfont,
    numbered=no,
    mdframed={
        linewidth=0pt,
        rightline=false, topline=false, bottomline=false,
        linecolor=black, backgroundcolor=black!2!white,
    },
    qed=\qedsymbol
]{thmproofbox}

\declaretheoremstyle[
    headfont=\normalfont,
    bodyfont=\normalfont,
    numbered=no,
    mdframed={
        linewidth=1pt,
        rightline=false,
        topline=false,
        bottomline=false,
        linecolor=black,
        backgroundcolor=black!2!white,
        skipabove=10pt,
        skipbelow=10pt
    },
    headpunct={}
]{stratbox}

\declaretheoremstyle[
    headfont=\normalfont,
    bodyfont=\normalfont,
    numbered=no,
    mdframed={
        linewidth=1pt,
        rightline=false,
        topline=false,
        bottomline=false,
        linecolor=black,
        skipabove=10pt,
        skipbelow=10pt
    },
    headpunct={}
]{coverbox}
\declaretheorem[style=coverbox, name={}]{cover}
\renewenvironment{proof}[1][\proofname]{\vspace{-10pt}\begin{cover}}{\end{cover}}

\declaretheorem[numberwithin=chapter,style=thmbox, name=Definition]{definition}
\declaretheorem[sibling=definition,style=thmbox, numbered=no, name=Example]{eg}
\declaretheorem[sibling=definition,style=thmbox, name=Proposition]{prop}
\declaretheorem[sibling=definition,style=thmbox, name=Theorem, numbered=yes]{theorem}
\declaretheorem[sibling=definition,style=thmbox, name=Lemma]{lemma}
\declaretheorem[sibling=definition,style=thmbox, name=Corollary]{corollary}

\declaretheorem[style=stratbox, name={}]{strat}
\renewenvironment{proof}[1][\proofname]{\vspace{-10pt}\begin{strat}}{\end{strat}}

\declaretheorem[style=thmproofbox, name=Proof]{replacementproof}
\renewenvironment{proof}[1][\proofname]{\vspace{-10pt}\begin{replacementproof}}{\end{replacementproof}}

\declaretheorem[style=thmbox, numbered=no, name=Note]{note}
\declaretheorem[style=thmbox, numbered=no, ]{temp}
\declaretheorem[style=thmtikz, numbered=no, name=.]{tikznt}

\newcommand{\bb}[1]{\mathbb{#1}}

\newcommand{\createcover}[3]{
    \begin{tabularx}{\textwidth}{|X|Y|}
        \hline
        \begin{minipage}[t]{0.6\textwidth}
                \cellcolor{black}
                \color{white}
                \vspace{2em} % Adjust the vertical space as needed
                \section{\Huge{\textbf{#1}}}
                \vfill
        \end{minipage} &
        \begin{minipage}[t]{0.35\textwidth}
            \vspace{4cm} % Adjust the vertical space as needed
            \large{#3}
            \vspace{#2}
            \vfill
        \end{minipage} \\
        \hline
    \end{tabularx}
}
