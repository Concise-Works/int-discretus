\subsection{Introduction to Sets}
\hspace*{1em}
In discrete math we work with some group of `things,' a thing or something
we fancily call an \textbf{object}. A group or categorization of objects is called a set.

\begin{theo}[a Set]{thm:set}
    Is a collection of objects.
\end{theo}

\noindent
\textbf{For Example:}
\begin{itemize}
    \item $S$ = The set of all students in a classroom.
    \item $A$ = The set of all vowels in the English alphabet.
    \item $\mathbb{Z}$ = The set of all integers.

\end{itemize}
Objects in a \textbf{set} are called \textbf{elements}.

\begin{theo}[an Element]{thm:element}
    An object that is a member of a given set.
\end{theo}

\noindent
To expand on the previous example:
\begin{itemize}
    \item $S = \{s_1, s_2, s_3\}$, where $s_1, s_2, s_3$ are students, i.e, elements of the set.
    \item $A = \{a, e, i, o, u\}$, where $a, e, i, o, u$ are vowels in the English alphabet.
    \item $\mathbb{Z} = \{\ldots, -3, -2, -1, 0, 1, 2, 3, \ldots\}$, elements of integer set.
\end{itemize}
We use curly braces to denote a set, and commas to separate elements.
The `$...$' or `dots' indicate that the set continues indefinitely in that direction.
When the set's pattern is clear to the reader, we use the dotted notation.
