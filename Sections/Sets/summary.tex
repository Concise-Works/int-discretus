\begin{greenbox}
    \textbf{Summary:} A \textbf{set} is a collection of `things' or objects. An object that is a member of a set is an
    \textbf{element} $4\in\mathbb{Z}$. In a set \textbf{order and
        repetition do not matter}. Sets can contain other sets, these subsections/slices/portions are
    called \textbf{subsets}. An \textbf{empty set} is denoted $\emptyset$ or $\{\}$. \textbf{Cardinality} is the element
    count of a set, it does not count sets beyond the top layer. $\{1, {2, 3}\}$ has a cardinality of 2, not 3.
    \textbf{Set-roster} notation is the explicit listing of a set like $\{...,1,2,3,...\}$. \textbf{Set-builder}
    notation is the description of a set $\{x\in \mathbb{Z} | x \textbf{ is even}\}$. Combining two sets is called a \textbf{union},
    the \textbf{intersection} is the common elements between two sets. The \textbf{cartesian product} of two sets is the set of all possible
    ordered pairs. The \textbf{difference} between sets $A$ and $B$ are elements in $A$ that are not in $B$. There
    are many more operations that we could discuss, but we will just stick to these for now.
\end{greenbox}

\begin{graybox}
    \textbf{Comments:} In or out of math, we talk about sets. In conversations,\\
    ``yeah that thing!" ``sorry, what thing?"
    Those people: are we talking about the group or a sub-group of those people. Will we make generalizations?
    Will we be specific? To draw connections, Humans $\in$
    Earth $\in$ Universe. Organizing files, sub-folder to sub-folder. Everything is a set of something.
\end{graybox}