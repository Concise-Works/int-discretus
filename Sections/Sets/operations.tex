\subsection{Set Operations}

The combination of $A=\{1,2,3\}$ and $B=\{a,b,c\}$ in order pairs are:
$$(1,a), (1,b), (1,c), $$
$$(2,a), (2,b), (2,c), $$
$$(3,a), (3,b), (3,c)$$

\noindent
Putting the above objects in a set yields the \textbf{Cartesian} product of $A$ and $B$.

% do Union, Intersection, Subraction

\begin{theo}[Cartesian]{thm:cartesian}
    The Cartesian product of two sets $A$ and $B$ is the set of all possible order
    pairs of elements from $A$ and $B$.\\\\
    Denoted: $A \times B$.
\end{theo}

\noindent
\textbf{For Example:}
\begin{itemize}
    \item $\{1, 2\} \times \{a, b\}$, then $A \times B = \{(1,a), (1,b), (2,a), (2,b)\}$.
    \item $A = \{1,2\}, B = \emptyset$, then $A \times B = \emptyset$. $B$ is empty, there is nothing to pair.
    \item $A = \{1\}, B = \{\emptyset\}$, then $A \times B = \{(1,\emptyset)\}$. $B$ contains 1 object to pair.
\end{itemize}
\textbf{Note Figure \ref{fig:empty_box}} from the previous section if the above caused confusion.


