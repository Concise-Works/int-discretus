\subsection{Set Operations}

% Subraction

Combining the two sets, $\{1,2,3\}$ and $\{a,b,c\}$, produce the set $\{1,2,3,a,b,c\}$, which
is called the \textbf{union}.\\

\begin{Def}[Union]{def:union}
    The set of elements that appear in either set $A$ or set $B$ is the union.\\
    Denoted: $A \cup B$.
\end{Def}

\noindent
This is also known as a \textbf{disjunction}, which is a fancy term for the word ``OR".\\

\noindent
\textbf{For Example:}
\begin{itemize}
    \item $\{1, 2\} \cup \{2, 3\} = \{1, 2, 3\}$.
    \item $\{1,2\} \cup \emptyset = \{1, 2\}$. There is nothing to add.
    \item $\{1\} \cup \{\emptyset\} = \{1, \emptyset\}$. The $\emptyset$ is an element in this case.
\end{itemize}

\noindent
The common elements of the two sets, $\{1,2,3\}$ and $\{2,3,4\}$, produce the set $\{2,3\}$,
the \textbf{intersection}.\\

\begin{Def}[Intersection]{def:intersection}
    The set of elements that appear in both sets $A$ and $B$ is the intersection.\\
    Denoted: $A \cap B$.
\end{Def}

\noindent
This is also known as a \textbf{conjunction}, which is a fancy term for the word ``AND".\\

\noindent
\textbf{For Example:}
\begin{itemize}
    \item $\{1, 2\} \cap \{2, 3\} = \{2\}$.
    \item $\{1\} \cap \{2\} = \emptyset$. There is nothing in common.
    \item $\{1\} \cap \emptyset = \emptyset$. There is nothing to compare.
\end{itemize}

\begin{Tip}
    To lessen the confusion between $\cup$ and $\cap$, think, ``$\cap$" for ``AND",\\
    since $\cap$ looks like a curved ``A" without the line. \\
\end{Tip}

\noindent
The combination of $A=\{1,2,3\}$ and $B=\{a,b,c\}$ in order pairs are:
$$(1,a), (1,b), (1,c), $$
$$(2,a), (2,b), (2,c), $$
$$(3,a), (3,b), (3,c)$$

\noindent
Putting the above objects in a set yields the \textbf{cartesian product} of $A$ and $B$.

\begin{Def}[Cartesian Product]{def:cartesian}
    The set of all possible order pairs of elements from sets $A$ and $B$.\\
    Denoted: $A \times B$.
\end{Def}

\noindent
\textbf{For Example:}
\begin{itemize}
    \item $\{1, 2\} \times \{a, b\} = \{(1,a), (1,b), (2,a), (2,b)\}$.
    \item $\{1,2\} \times \emptyset = \emptyset$. There is nothing to pair.
    \item $\{1\} \times \{\emptyset\} = \{(1,\emptyset)\}$. The $\emptyset$ is an element in this case.
\end{itemize}

\begin{Note}
    \textbf{Note:} Visit `\textbf{Figure \ref{fig:empty_box}}' in the previous section if $\emptyset$ causes confusion.
\end{Note}


\noindent
We have sets $A=\{1,2,3\}$ and $B=\{2,3,4\}$, to take all in $A$ that is not in $B$, i.e.,
items that are unique to $A$, yields the set $\{1\}$, the \textbf{difference}.\\

\begin{Def}[Difference]{def:difference}
    The set of all elements that are in set $A$ but not in set $B$ is the difference.\\
    Denoted: $A - B$.
\end{Def}

\noindent
\textbf{For Example:}
\begin{itemize}
    \item $\{1, 2\} - \{2, 3\} = \{1\}$.
    \item $\{1\} - \{1\} = \emptyset$.
    \item $\{1,2\} - \emptyset = \{1, 2\}$. There is nothing to remove.
\end{itemize}

\noindent
All possible subsets of a set is called the \textbf{power set}.
\begin{Def}[Power Set]{def:power}
    The set of all subsets of a set.\\
    Denoted: $\mathcal{P}(A)$, $A$ is a set.
\end{Def}

\noindent
\textbf{For Example:}
\begin{itemize}
    \item $\mathcal{P}(\emptyset) = \{\emptyset\}$.
    \item $\mathcal{P}(\{1\}) = \{\emptyset, \{1\}\}$.
    \item $\mathcal{P}(\{1, 2\}) = \{\emptyset, \{1\}, \{2\}, \{1, 2\}\}$.
    \item $\mathcal{P}(\{1, 2, 3\}) = \{\emptyset, \{1\}, \{2\}, \{3\}, \{1, 2\}, \{1, 3\}, \{2, 3\}, \{1, 2, 3\}\}$.
\end{itemize}

\noindent
The number of subsets grow exponentially with the number of elements in the set.
The rate it grows is $2^{|n|}$, where $n$ is the number of elements in the set.

\begin{Def}[Power Set Cardinality]{def:pow_set_card}
    The cardinality of the power set grows $2^{|n|}$, where $n$ is the number of elements in the set.
\end{Def}

\noindent
There are more operations that we could discuss, but we will stop here for now.
I encourage you challenge these definitions, create different cases, and test them.\\
