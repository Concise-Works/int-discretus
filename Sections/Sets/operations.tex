\subsection{Set Operations}

% Subraction

Combining the two sets, $\{1,2,3\}$ and $\{a,b,c\}$, produce the set $\{1,2,3,a,b,c\}$, which is
is called the \textbf{union} of $A$ and $B$.\\

\begin{theo}[Union]{thm:union}
    The union of two sets $A$ and $B$ is the set of all elements that are in both sets.\\\\
    Denoted: $A \cup B$.
\end{theo}

\noindent
This is also known as a \textbf{disjunction}, which is a fancy term for the word ``OR".\\

\noindent
\textbf{For Example:}
\begin{itemize}
    \item $\{1, 2\} \cup \{2, 3\} = \{1, 2, 3\}$.
    \item $\{1,2\} \cup \emptyset = \{1, 2\}$. There is nothing to add.
    \item $\{1\} \cup \{\emptyset\} = \{1, \emptyset\}$. The empty set is an object.
\end{itemize}

\noindent
The common elements of the two sets, $\{1,2,3\}$ and $\{2,3,4\}$, produce the set $\{2,3\}$,
the \textbf{intersection}.\\

\begin{theo}[Intersection]{thm:intersection}
    The intersection of two sets $A$ and $B$ is the set of all elements that are in both sets.\\\\
    Denoted: $A \cap B$.
\end{theo}

\noindent
This is also known as a \textbf{conjunction}, which is a fancy term for the word ``AND".\\

\textbf{For Example:}
\begin{itemize}
    \item $\{1, 2\} \cap \{2, 3\} = \{2\}$.
    \item $\{1,2\} \cap \emptyset = \emptyset$. There is nothing to compare.
    \item $\{1\} \cap \{\emptyset\} = \emptyset$. The empty set is not an object.
\end{itemize}

\noindent
\textbf{Tip:} To lessen the confusion between $\cup$ and $\cap$, think, ``$\cap$" for ``AND",\\
since $\cap$ looks like an ``A" without the line.\\

\noindent
The combination of $A=\{1,2,3\}$ and $B=\{a,b,c\}$ in order pairs are:
$$(1,a), (1,b), (1,c), $$
$$(2,a), (2,b), (2,c), $$
$$(3,a), (3,b), (3,c)$$

\noindent
Putting the above objects in a set yields the \textbf{Cartesian Product} of $A$ and $B$.

\begin{theo}[Cartesian Product]{thm:cartesian}
    The Cartesian product of two sets $A$ and $B$ is the set of all possible order
    pairs of elements from $A$ and $B$.\\\\
    Denoted: $A \times B$.
\end{theo}

\noindent
\textbf{For Example:}
\begin{itemize}
    \item $\{1, 2\} \times \{a, b\} = \{(1,a), (1,b), (2,a), (2,b)\}$.
    \item $\{1,2\} \times \emptyset = \emptyset$. $B$ is empty, there is nothing to pair.
    \item $\{1\} \times \{\emptyset\} = \{(1,\emptyset)\}$. $B$ contains 1 object to pair.
\end{itemize}
\textbf{Note Figure \ref{fig:empty_box}} from the previous section if the above caused confusion.\\

\noindent
We have sets $A=\{1,2,3\}$ and $B=\{2,3,4\}$, if we remove the common elements
$A$ has with $B$, i.e., take all in $A$ that is not in $B$, we get the set $\{1\}$, the \textbf{difference}.\\

\begin{theo}[Difference]{thm:difference}
    The difference of two sets $A$ and $B$ is the set of all elements that are in $A$ but not in $B$.\\\\
    Denoted: $A - B$.
\end{theo}

\noindent
\textbf{For Example:}
\begin{itemize}
    \item $\{1, 2\} - \{2, 3\} = \{1\}$.
    \item $\{1,2\} - \emptyset = \{1, 2\}$. There is nothing to remove.
    \item $\{1\} - \{\emptyset\} = \{1\}$. The empty set is not an object.
\end{itemize}


