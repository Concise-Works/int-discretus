\vspace{2em}
\begin{greenbox}
    \begin{spacing}{1.5}
        \textbf{Summary: } A \textbf{function} is a \textbf{relationship} between two sets. A set for which we use
        as an input, and a set that will house our outputs. A relationship can be described in
        ordered pairs, take sets $A$ and $B$, over some relation $R$, $x\in A$ relates to $y\in B$
        such that $(x,y)\in R$. The ordered pairs in a \underline{relation is a subset of $A\times B$},
        which includes the emptyset.\\

        \noindent
        A function takes in one input and produces one output. The set of all our inputs is called the
        \textbf{domain}, the set of our outputs the \textbf{codomain}. The set of all possible mappings
        from the domain to the codomain is called the \textbf{range}.\\

        \noindent
        If elements in the codomain all have a mapping, the function is \textbf{onto} or\\
        \textbf{surjective}. If elements in the codomain have a unique mapping, the function is
        \textbf{one-to-one} or \textbf{injective}.
        If a function is both \textbf{onto} and \textbf{one-to-one}, it is \textbf{bijective}.
    \end{spacing}
\end{greenbox}
