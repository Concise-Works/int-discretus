\subsection{Set Quantifiers}
Set quantifiers help describe particular members of a set. Let's examine \underline{the set of
    all musical artists.}\\

\noindent
Observe the claim:

\begin{center}
    \Large
    \textit{``Artists that live in texas, make country music."}
\end{center}

\noindent
This claim generalizes an entire group of artist. Perhaps not all
Texan artists make country music. This is called a {\textbf{universal generalization}}.\\

\noindent
Using Set-Builder notation:
\begin{itemize}
    \item $A$, is the set of all artists.
    \item $T(x)$, returns true if $x$ is from Texas.
    \item $C(x)$, returns true if $x$ makes country music.
    \item Yielding, ``for all $x\in A \mid T(x) \rightarrow C(x)$.''
    \item Reads, \underline{``for all $x$ in $A$, if $x$ is from Texas, then $x$ makes country music.''}
\end{itemize}

\noindent
The symbol for \underline{``For All'' is ``$\forall$''}, an upside-down ``A'', which gives us:

\begin{center}
    \Large
    ``$\forall x \in A \mid T(x) \rightarrow C(x)$.''
\end{center}

\newpage

\begin{Def}[Universal Generalization]{def:universal}
    A claim that applies to all elements in a set.
    Denoted: $\forall x \in A \mid P(x) \rightarrow Q(x)$, given a set $A$ and predicates $P(x)$ and $Q(x)$.
\end{Def}

\noindent
Observe the claim:

\begin{center}
    \Large
    \textit{``There exists an artist that lives in Texas, that makes dubstep."}
\end{center}

\noindent
Here we describe a particular artist. This is called an {\textbf{existential instantiation}}.\\

\noindent
Using Set-Builder notation:
\begin{itemize}
    \item $A$, is the set of all artists.
    \item $T(x)$, returns true if $x$ is from Texas.
    \item $D(x)$, returns true if $x$ makes dubstep.
    \item Yielding, ``there exists an $x\in A \mid T(x) \land D(x)$.''
    \item Reads, \underline{``there exists an $x$ in $A$, such that $x$ is from Texas and makes dubstep.''}
\end{itemize}

\noindent
The symbol for \underline{``There Exists'' is ``$\exists$''}, an backwards ``E'', which gives us:

\begin{center}
    \Large
    ``$\exists x \in A \mid T(x) \land D(x)$.''\\
\end{center}

\begin{Def}[Existential Instantiation]{def:existential}
    A claim that applies to at least one element in a set.\\
    Denoted: $\exists x \in A \mid P(x) \land Q(x)$, given a set $A$ and predicates $P(x)$ and $Q(x)$.
\end{Def}

\noindent
An existential claim creates the set of all elements that satisfy the claim. So there \textit{could}
exist multiple artists that live in Texas and make dubstep.

\noindent
\subsection*{When to use $\rightarrow$ vs. $\land$:}
\begin{itemize}
    \item \textbf{Universal Generalization:} Uses ``$\rightarrow$'' to imply a relationship.
    \item \textbf{Existential Instantiation:} Uses ``$\land$'' to describe qualities of a particular member.
\end{itemize}

\newpage

\subsection{Nested Quantifiers}

\noindent
Consider the claim:

\begin{center}
    \Large
    \textit{``Every artist in a record label, has someone else as their manager."}
\end{center}

\begin{itemize}
    \item $P$, the set of all people.
    \item $A(x)$, $x$ is an artist.
    \item $R(x)$, $x$ has a record label.
    \item $M(x, y)$, $y$ is $x$'s manager.
\end{itemize}

\noindent
We are dealing with two predicates, someone who is $x$ and someone who is $y$. Breaking it down:
\begin{itemize}
    \item $E$ = Every artist in a record label = $\forall x \in P \mid A(x) \land R(x)$.
    \item $S$ = Said artist has a manager =  $\exists y \in P \mid M(x, y)$.
    \item $E$ implies $S$, so $E \rightarrow S$.
    \item Yielding, \underline{$(\forall x \in P \mid A(x) \land R(x)) \rightarrow (\exists y \in P \mid M(x, y))$.}
\end{itemize}

\noindent
Our final statement reads:

\begin{center}
    \Large
    ``For all $x$ in $P$, if $x$ is an artist and has a record label, then there exists a $y$ in $P$ that is $x$'s manager.''
\end{center}

\noindent
Or concisely:\\

\begin{center}
    \Large
    ``For every x there exists y in P, if x is an artist and has a record label, then y is x's manager.''
\end{center}

\noindent
Written:\\

\begin{center}
    \Large
    $\forall x \exists y \in P \mid (A(x)\land R(x)) \rightarrow M(x,y)$
\end{center}

\noindent
If the set is clear, we can omit syntax and write:\\

\begin{center}
    \Large
    $\forall x \exists y \mid (A(x)\land R(x)) \rightarrow M(x,y)$
\end{center}

\newpage

\noindent
Observe the claim:

\begin{center}
    \Large
    \textit{``There was exactly one person who was late to the meeting.''}
\end{center}


Assuming the set of all people:
\begin{itemize}
    \item $L(x)$, $x$ was late to the meeting.
    \item Someone was late = $\exists x \mid L(x)$.
\end{itemize}

To be the only person, means \underline{on one else was late.} so:
\begin{itemize}
    \item No one else was late = $\forall y\mid \neg L(y)$.
    \item however $x$ could equal $y$ so we add, $\forall y\mid (( x\neq y) \rightarrow \neg L(y)$).
\end{itemize}

\begin{Note}
    \textbf{Note:} Remember to use $\rightarrow$ to imply a relationship. To use $\forall$ is to describe a universal claim.
\end{Note}

\noindent
Taking our Existential claim and adding the Universal claim, we get:

\begin{center}
    \Large
    $(\exists x \mid L(x)) \land (\forall y\mid (( x\neq y) \rightarrow \neg L(y)))$
\end{center}

\noindent
Or concisely:

\begin{center}
    \Large
    $\exists x \forall y\mid L(x) \land (( x\neq y) \rightarrow \neg L(y))$

\end{center}

\noindent
Assuming monogamy, observe the claim:

\begin{center}
    \Large
    \textit{``Every person who is married, is married to one other person.''}
\end{center}


Assuming the set the set of all people:
\begin{itemize}
    \item $R(x), x$ is married.
    \item $M(x, y), x$ is married to $y$.
    \item Everyone who is married = $\forall x \mid R(x)$.
    \item Married to one other person = $\exists y \mid ((x\neq y) \land M(x, y))$.
\end{itemize}

\noindent
We have the statement:

\begin{center}
    \Large
    $(\forall x \mid R(x)) \rightarrow (\exists y \mid ((x\neq y) \land M(x, y)))$
\end{center}

\noindent
Or:

\begin{center}
    \Large
    $\forall x \exists y \mid R(x) \rightarrow ((x\neq y) \land M(x, y))$
\end{center}