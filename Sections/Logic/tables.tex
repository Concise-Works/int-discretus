\subsection{Truth Tables (CNF \& DNF)}
Writing whole sentences in logic can be cumbersome even in their reduced logical forms.
So we use \textbf{truth tables} to help keep track our evaluations.\\

\noindent
To demonstrate, let's use a table to show the values $P$ and $\neg P$ Our goal is to
\underline{find all possible values} of $P$ to then evaluate $\neg P$.

\begin{center}
    \begin{tabular}{|c|c|}
        \hline
        \rowcolor{OliveGreen!10}
        $P$ & $\neg P$ \\
        \hline
        T   & F        \\
        F   & T        \\
        \hline
    \end{tabular}
\end{center}
\noindent
$T$ = ``True'' and $F$ = ``False'', we refer to these truth values as \underline{\textbf{booleans}}.\\


\begin{Def}[Boolean]{def:boolean}
    A Boolean is a value that can only be either true or false.
\end{Def}

\newpage

\noindent
Our goal here is to find all possible value combinations of $P$ and $Q$ to then evaluate $P \land Q$ and $P \lor Q$.\\


\begin{center}
    \begin{tabular}{|c|c|c|c|}
        \hline
        \rowcolor{OliveGreen!10}
        $P$ & $Q$ & $P \land Q$ & $P \lor Q$ \\
        \hline
        T   & T   & T           & T          \\
        T   & F   & F           & T          \\
        F   & T   & F           & T          \\
        F   & F   & F           & F          \\
        \hline
    \end{tabular}
\end{center}

\noindent
We will read this table, $P$ then $Q$ boolean values respectively:
\begin{enumerate}
    \item $T$ then $T$, So $T\land T = T$, and $T\lor T = T$.
    \item $T$ then $F$, So $T\land F = F$, and $T\lor F = T$.
    \item $F$ then $T$, So $F\land T = F$, and $F\lor T = T$.
    \item $F$ then $F$, So $F\land F = F$, and $F\lor F = F$.
\end{enumerate}

\begin{Tip}
    When working with new concepts, ask yourself, what was going
    through the creator's mind. Why this way and not some other way? Try
    to understand the intuition behind an idea rather than pure memorization.\\
\end{Tip}

\noindent
Now lets evaluate $\neg(P) \lor \neg(Q)$, \underline{the full table will be on the next page:}\\

\begin{center}
    \begin{tabular}{|c|c|c|c|c|c|}
        \hline
        \rowcolor{OliveGreen!10}
        $P$ & $Q$ & $\neg P$ & $\neg Q$ & $\neg P \lor \neg Q$ \\
        \hline
        T   & T   &          &          &                      \\
        T   & F   &          &          &                      \\
        F   & T   &          &          &                      \\
        F   & F   &          &          &                      \\
        \hline
    \end{tabular}
\end{center}

\noindent
Try to think and \underline{fill this out on your own}, then compare your answers with the next page.\\

\newpage

\noindent
This is the complete table for $\neg(P) \lor \neg(Q)$:
\begin{center}
    \begin{tabular}{|c|c|c|c|c|c|}
        \hline
        \rowcolor{OliveGreen!10}
        $P$ & $Q$ & $\neg P$ & $\neg Q$ & $\neg P \lor \neg Q$ \\
        \hline
        T   & T   & F        & F        & F                    \\
        T   & F   & F        & T        & T                    \\
        F   & T   & T        & F        & T                    \\
        F   & F   & T        & T        & T                    \\
        \hline
    \end{tabular}
\end{center}

\vspace{1em}
\noindent
Now let's try to evaluate $P \land \neg Q \lor W$. The filled table will be on the \underline{next page}.

\begin{center}
    \begin{tabular}{|c|c|c|c|c|c|c|}
        \hline
        \rowcolor{OliveGreen!10}
        $P$ & $Q$ & $W$ & $\neg Q$ & $P \land \neg Q$ & $P \land \neg Q \lor W$ \\
        \hline
        T   & T   & T   &          &                  &                         \\
        T   & T   & F   &          &                  &                         \\
        T   & F   & T   &          &                  &                         \\
        T   & F   & F   &          &                  &                         \\
        F   & T   & T   &          &                  &                         \\
        F   & T   & F   &          &                  &                         \\
        F   & F   & T   &          &                  &                         \\
        F   & F   & F   &          &                  &                         \\
        \hline
    \end{tabular}
\end{center}

\noindent
\underline{\textbf{Notice} a pattern is starting to begin with our variables,} $P$, $Q$, and $W$. To get
all possible combinations, we increase the periods of repeating true and false.
\begin{itemize}
    \item With $W$, $T$ and $F$ alternate each row.
    \item For $Q$ we alternate every 2 rows.
    \item For $P$ we alternate every 4 rows.
    \item 4 variables: every 16 rows, 5: every 32, and so on.
\end{itemize}
The pattern is \underline{$2^n$ where $n$ is the number of variables.}\\

\begin{Def}[Growth of Truth Table]{def:table_growth}
    The number of rows in a table grow $2^n$ where $n$ is the number of variables.
\end{Def}

\newpage

\noindent
This is the complete table for $P \land \neg Q \lor W$:
\begin{center}
    \begin{tabular}{|c|c|c|c|c|c|c|}
        \hline
        \rowcolor{OliveGreen!10}
        $P$ & $Q$ & $W$ & $\neg Q$ & $P \land \neg Q$ & $P \land \neg Q \lor W$ \\
        \hline
        T   & T   & T   & F        & F                & T                       \\
        T   & T   & F   & F        & F                & F                       \\
        T   & F   & T   & T        & T                & T                       \\
        T   & F   & F   & T        & T                & T                       \\
        F   & T   & T   & F        & F                & T                       \\
        F   & T   & F   & F        & F                & F                       \\
        F   & F   & T   & T        & F                & T                       \\
        F   & F   & F   & T        & F                & F                       \\
        \hline
    \end{tabular}
\end{center}
\noindent
\textbf{Truth tables are functions}, given a set of inputs, it will produce an output. Our
above table could be written as the function $f(P, Q, W) = P \land \neg Q \lor W$.\\

\noindent
We can leverage tables to derive new functions. Observe the table below:\\

\begin{center}
    \begin{tabular}{|c|c|c|}
        \hline
        \rowcolor{OliveGreen!10}
        $P$ & $Q$ & $f(x)$ \\
        \hline
        T   & T   & F      \\
        \rowcolor{purple!10}
        T   & F   & T      \\
        F   & T   & F      \\
        \rowcolor{purple!10}
        F   & F   & T      \\
        \hline
    \end{tabular}
\end{center}

\noindent
Note the highlighted rows, our function is true if:
\begin{center}
    \large
    ``$P$ is true and $Q$ is false'' or ``$P$ is false and $Q$ is false.''\\
\end{center}

\noindent
Let's convert our statements into propositional logic with our boolean operators:
\begin{itemize}
    \item ``$P$ is true and $Q$ is false'' = $P \land \neg Q$.
    \item ``$P$ is false and $Q$ is false.'' = $\neg P \land \neg Q$.
    \item Adding back our ``or'' yields: $(P \land \neg Q) \lor (\neg P \land \neg Q)$.
\end{itemize}

\begin {Note}
\textbf{Note:} $P$ and $Q$ are propositions, and propositions are claims that we assert to be true.
To say ``$P$ is false,'' is to say ``not $P$ is true,'' i.e, ``$\neg P$''.
\end{Note}

\newpage

\noindent
Known as \textbf{disjunctive normal form (DNF)}, where we take statements held by
``AND'' and join them by ``OR.'' For reference: $(P \land \neg Q) \lor (\neg P \land \neg Q)$:

\begin{Def}[Disjunctive Normal Form (DNF)]{def:dnf}
    Statements held by ``AND'' joined by ``OR,'' i.e., conjunctions joined by disjunctions.
\end{Def}

\noindent
We can also derive a function by negating false rows, which also holds true:\\
\begin{center}
    \begin{tabular}{|c|c|c|}
        \hline
        \rowcolor{OliveGreen!10}
        $P$ & $Q$ & $f(x)$ \\
        \hline
        \rowcolor{purple!10}
        T   & T   & F      \\
        T   & F   & T      \\
        \rowcolor{purple!10}
        F   & T   & F      \\
        F   & F   & T      \\
        \hline
    \end{tabular}
\end{center}
\begin{center}
    \large
    ``not row one'' and ``not row three.''\\
    ``$\neg(P \land Q)$'' and ``$\neg(\neg P \land Q)$''.
\end{center}

\noindent
To simplify, we will use \underline{\textbf{De Morgan's Laws}.}\\

\begin{Def}[De Morgan's Laws]{def:demorgan}
    For booleans $P$ and $Q$:\\
    \textbf{First Law:} $\neg(P \land Q) = \neg P \lor \neg Q$.\\
    \textbf{Second Law:} $\neg(P \lor Q) = \neg P \land \neg Q$.\\

    \noindent
    i.e., negation distributes, $\lor$ to $\land$ and $\land$ to $\lor$.
\end{Def}
And we'll need the \underline{\textbf{law of double negation}}.\\

\begin{Def}[Double Negation]{def:double_negation}
    For a boolean $P$, $\neg(\neg P) = P$.
\end{Def}

\newpage

\noindent
To simplify ``$\neg(P \land Q)\land \neg(\neg P \land Q)$'' we well employ both techniques in a \underline{\textbf{two column proof}:}\\

\begin{tabular}{r m{.25mm} ll}
    $\neg(P \land Q)$                               & $\land$ & $\neg(\neg P \land Q)$                              & \text{ Given}            \\
    \cellcolor{OliveGreen!10}$(\neg P \lor \neg Q)$ & $\land$ & $\neg(\neg P \land Q)$                              & \text{ De Morgan's Laws} \\
    $(\neg P \lor \neg Q)$                          & $\land$ & \cellcolor{OliveGreen!10}$(\neg\neg P \lor \neg Q)$ & \text{ De Morgan's Laws} \\
    $(\neg P \lor \neg Q)$                          & $\land$ & \cellcolor{OliveGreen!10}$(P \lor \neg Q)$          & \text{ Double Negation}  \\
\end{tabular}

\vspace{1em}
\noindent
\textbf{Yielding}: \underline{$(\neg P \lor \neg Q) \land (P \lor \neg Q)$}.\\

\noindent
This is called \textbf{conjunctive normal form (CNF)}, where we take statements held by
``OR'' and join them by ``AND.''

\begin{Def}[Conjunctive Normal Form (CNF)]{def:cnf}
    Statements held by ``OR'' joined by ``AND,'' i.e., disjunctions joined by conjunctions.
\end{Def}

\noindent
\underline{\textbf{Any boolean function} can be written in DNF or CNF}. Using a table helps
derive these forms.\\