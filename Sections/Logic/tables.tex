\subsection{Truth Tables}
Writing whole sentences in logic can be cumbersome even in their reduced logical forms.
So we use \textbf{truth tables} to help keep track our evaluations.\\

\noindent
To demonstrate, let's use a table to show the values $P$ and $\neg P$\\
Our goal is to \underline{find all possible values} of $P$ to then evaluate $\neg P$.\\
Let's take that concept and evaluate $P$ and $Q$ for ``$P \land Q$'' and ``$P \lor Q$.''\\

\begin{center}
    \begin{tabular}{|c|c|}
        \hline
        \rowcolor{OliveGreen!10}
        $P$ & $\neg P$ \\
        \hline
        T   & F        \\
        F   & T        \\
        \hline
    \end{tabular}
\end{center}
\noindent
$T$ = ``True'' and $F$ = ``False'', we refer to these truth values as \underline{\textbf{booleans}}.\\


\begin{Def}[Boolean]{def:boolean}
    A Boolean is a value that can only be either true or false.
\end{Def}

\newpage

\noindent
Our goal here is to find all possible value combinations of $P$ and $Q$ to then evaluate $P \land Q$ and $P \lor Q$.\\


\begin{center}
    \begin{tabular}{|c|c|c|c|}
        \hline
        \rowcolor{OliveGreen!10}
        $P$ & $Q$ & $P \land Q$ & $P \lor Q$ \\
        \hline
        T   & T   & T           & T          \\
        T   & F   & F           & T          \\
        F   & T   & F           & T          \\
        F   & F   & F           & F          \\
        \hline
    \end{tabular}
\end{center}

We will read this table's $P$ then $Q$ boolean values respectively:
\begin{enumerate}
    \item $T$ then $T$, So $T\land T = T$, and $T\lor T = T$.
    \item $T$ then $F$, So $T\land F = F$, and $T\lor F = T$.
    \item $F$ then $T$, So $F\land T = F$, and $F\lor T = T$.
    \item $F$ then $F$, So $F\land F = F$, and $F\lor F = F$.
\end{enumerate}

\begin{Tip}
    When working with new concepts, ask yourself what was going through the creator's mind.
    Try to understand the logic and intuition behind concepts rather than memorizing them.
    So when you forget, you'll be able to derive the concept from scratch, or consider a new novel idea.\\
\end{Tip}