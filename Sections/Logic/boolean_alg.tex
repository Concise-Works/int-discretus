
\subsection{Boolean Algebra\\ \& Logical Equivalences}
\vspace{1em}

In digital systems we manipulate binary values to evaluate logic. These systems use
states of \underline{``on'' and ``off'' to represent 1 (true) and 0 (false).}\\

\noindent
This is identical to the propositional logic we've been discussing. Observe:\\
\begin{center}
    \begin{tabular}{c c c}
        Boolean Algebra &     & Propositional Logic \\
        \hline
        $\bar{x}$       & NOT & $\neg x$            \\
        $x \cdot y$     & AND & $x \land y$         \\
        $x + y$         & OR  & $x \lor y$          \\
        $x \oplus y$    & XOR & $x \oplus y$        \\
        \hline
    \end{tabular}
\end{center}

\begin{Note}
    \textbf{Note:} XOR in programming is often represented by the caret symbol `$^\wedge $'.
\end{Note}

\noindent
This is the basis for logic gates and circuits, of which we will not discuss here.\\

\newpage

\noindent
Lets use the following truth table to demonstrate boolean algebra conversion:\\
\begin{center}
    \begin{tabular}{|c|c|c|}
        \hline
        \rowcolor{OliveGreen!10}
        $P$ & $Q$ & $f(x)$ \\
        \hline
        T   & T   & F      \\
        T   & F   & T      \\
        F   & T   & F      \\
        F   & F   & T      \\
        \hline
    \end{tabular}
\end{center}

\begin{tabular}{cc|l}
    \textbf{DNF}                                  & \textbf{CNF}                                 &                     \\
    \hline
    $(P \land \neg Q) \lor (\neg P \land \neg Q)$ & $(\neg P \lor \neg Q) \land (P \lor \neg Q)$ & Propositional Logic \\
    $(P \cdot \bar{Q}) + (\bar{P} \cdot \bar{Q})$ & $(\bar{P} + \bar{Q}) \cdot (P + \bar{Q})$    & Boolean Algebra     \\
    \hline
\end{tabular}

\vspace*{2em}

\noindent
We say \underline{$(P\bar{Q}) + (\bar{P} \bar{Q})$ is \textbf{equivalent} to $(\bar{P} + \bar{Q}) (P + \bar{Q})$.}
This doesn't mean they are syntactically the same or re-arrangeable to match. Rather that they evaluate to the same truth values.\\

\noindent
I.e, we could swap out $f(x)$ for either of the two expressions and \underline{the truth table remains the same.}\\

\begin{Note}
    \textbf{Note:} the dot refers to multiplication so ``$P\cdot Q$'' is ``$PQ$'', they are one and the same.
\end{Note}

\begin{Def}[Logical Equivalence]{def:log_equiv}
    Two expressions are equivalent if they evaluate to the same truth values.\\
    Denoted: $P \equiv Q$.
\end{Def}

\noindent
Obverse the claim:
\begin{center}
    \Large
    \textit{``If you don't study, then you won't pass.''}
\end{center}

This is an example of a \textbf{conditional statement}. Let's break the statement down:
\begin{itemize}
    \item \textbf{S} to study.
    \item \textbf{P} to pass.
\end{itemize}

\noindent
The statement says, ``$\neg S$ \textbf{implies} $\neg P$'', i.e.,
``$\neg S$ \textbf{then} $\neg P$'', formally written as ``$\neg S \rightarrow \neg P$.''\\

\newpage

\noindent
This type of statement is called an \textbf{implication}.
\begin{Def}[Implication]{def:imp}
    A conditionally statement of the form, if $P$ then $Q$.\\
    Denoted: $P \rightarrow Q$.
\end{Def}

\noindent
Observe the following truth table for the implication:

\begin{center}
    \begin{tabular}{|c|c|c|}
        \hline
        \rowcolor{OliveGreen!10}
        $P$ & $Q$ & $P \rightarrow Q$ \\
        \hline
        T   & T   & T                 \\
        T   & F   & F                 \\
        F   & T   & T                 \\
        F   & F   & T                 \\
        \hline
    \end{tabular}
\end{center}

\noindent
Think of the implication as holding a promise:
\begin{itemize}
    \item If the promise to do something, and it gets done, I held my promise (true).
    \item If I promise to do something, and it doesn't get done, I broke my promise (false).
    \item If I never promised to do anything, then I can't break my promise (true).
\end{itemize}
The last statement is true because \underline{there was no promise to break,} hence, it's \textbf{Vacuously True}.\\
Likewise, you cannot deny my claim if I never made one.\\

\begin{Note}
    \textbf{Note:} We saw a \textbf{vacuously true} statement before when saying ``$\emptyset$ is a subset of all sets.''
    As it's impossible to deny that nothing is a part of something.
\end{Note}

\noindent
It actually turns out $P \rightarrow Q$ is equivalent to $\neg P \lor Q$. Observe the table below:\\

\begin{center}
    \begin{tabular}{|c|c|c|c|c|}
        \hline
        \rowcolor{OliveGreen!10}
        $P$ & $Q$ & $\neg P$ & $\neg P \lor Q$ & $P \rightarrow Q$ \\
        \hline
        T   & T   & F        & T               & T                 \\
        T   & F   & F        & F               & F                 \\
        F   & T   & T        & T               & T                 \\
        F   & F   & T        & T               & T                 \\
        \hline
    \end{tabular}
\end{center}

{\centering $P \rightarrow Q\equiv\neg P \lor Q$.\\}

\newpage

\noindent
This is known as the \underline{\textbf{conditional identity}}. There are many more logical equivalences
that we will need on our journey. Reference the table below, \textbf{(This will be your best friend)}:\\

\vspace{1em}
\noindent
{\Large \textbf{Logical Equivalences:}\\}

\noindent
\begin{tabular}{|p{2cm}|l|l|}
    \hline
    \cellcolor{OliveGreen!10}Idempotent:                                             & $p\lor p\equiv p$                                   & $p\land p\equiv p$                                   \\
    \hline
    \cellcolor{OliveGreen!10}Associative:                                            & $(p\lor q) \lor r \equiv p \lor (q \lor r)$         &
    $(p\land q) \land r \equiv p \land (q\land r)$                                                                                                                                                \\
    \hline
    \cellcolor{OliveGreen!10}Commutative:                                            & $p\lor q\equiv q \lor p$                            & $p\land q \equiv q \land p$                          \\
    \hline
    \cellcolor{OliveGreen!10}Distributive:                                           & $p\lor (q\land r) \equiv (p\lor q) \land (p\lor r)$ & $p\land (q\lor r) \equiv (p\land q) \lor (p\land r)$ \\
    \hline
    \cellcolor{OliveGreen!10}Identity:                                               & $p\lor F \equiv p$                                  & $p\land T\equiv p$                                   \\
    \hline
    \cellcolor{OliveGreen!10}Domination:                                             & $p\land F \equiv F$                                 & $p\lor T\equiv T$                                    \\
    \hline
    \cellcolor{OliveGreen!10}Double Negation:                                        & \multicolumn{2}{l|}{$\neg\neg p\equiv p$}                                                                  \\
    \hline
    \cellcolor{OliveGreen!10}\multirow{2}{*}                                         & $p\land \neg p \equiv F$                            & $p\lor \neg p \equiv T$                              \\
    \cellcolor{OliveGreen!10}\raisebox{.8\normalbaselineskip}[0pt][0pt]{Complement:} & $\neg T \equiv F$                                   & $\neg F \equiv T$                                    \\
    \hline
    \cellcolor{OliveGreen!10}De Morgan's Laws:                                       & $\neg(p\lor q) \equiv \neg p \land \neg q$          &
    $\neg(p\land q) \equiv \neg p \lor \neg q$
    \\
    \hline
    \cellcolor{OliveGreen!10}Absorption:                                             & $p\lor (p\land q) \equiv p $                        & $p\land (p\lor q) \equiv p$                          \\
    \hline
    \cellcolor{OliveGreen!10}Conditional identities:                                 & $p\rightarrow q \equiv \neg p \lor q$               &
    $p\leftrightarrow q \equiv (p\rightarrow q)\land (q\rightarrow p)$                                                                                                                            \\
    \hline
\end{tabular}

\begin{Tip}
    Please, \underline{\textbf{do not}} try to memorize all of these, they will come naturally with practice. If
    you are taking a class, it's definitely not worth the mental effort. Do practice problems that
    require you to think about the rules.\\
\end{Tip}

\noindent
Observe the statement:

\begin{center}
    \Large
    \textit{``If and only if you study, then you will pass.''}
\end{center}

\noindent
This is a \textbf{biconditional statement}. For ``$S$ = study'' and ``$P$ = pass'':\\
Written ``$S \leftrightarrow P$'', which means ``$S\rightarrow P$ and $P\rightarrow S$''.\\

\begin{Note}
    \textbf{Note:} In writting, you may also see ``iff'', which is shorthand for ``if and only if''.\\
\end{Note}

\newpage

\noindent
To define the biconditional statement:
\begin{Def}[Biconditional]{def:bicond}
    A biconditional statement is true if and only if both propositions imply each other.\\
    Denoted as: $P \leftrightarrow Q$, of form, $P\rightarrow Q\land Q\rightarrow P$.\\
\end{Def}

\noindent
This is an important distinction to make as $P\rightarrow Q$ does not mean $Q\rightarrow P$.\\
Using the conditional identity reveals this, $\neg P \lor Q \not\equiv \neg Q \lor P$:\\

\begin{center}
    \begin{tabular}{|c|c|c|c|c|}
        \hline
        \rowcolor{OliveGreen!10}
        $P$ & $Q$ & $\neg P \lor Q$ & $\neg Q \lor P $ & $P\leftrightarrow Q$ \\
        \hline
        T   & T   & T               & T                & T                    \\
        T   & F   & F               & T                & F                    \\
        F   & T   & T               & F                & F                    \\
        F   & F   & T               & T                & T                    \\
        \hline
    \end{tabular}
\end{center}

\noindent
Now we can evaluate if two expressions are equivalent. Let's prove this next example:\\

\noindent
\textbf{\textit{Prove:}} $((p \land q) \lor (p \land s)) \lor (p \land r) \equiv p \land ((q \lor r) \lor s)$ \\

\noindent
The right-hand expression is simpler, it'll be more intuitive to expand it and match.\\
\noindent
% \textbf{\textit{Proof:}} $P
\begin{center}
    \begin{tabular}{l l m{.1mm} l m{.1mm} l l}
        1. &                                          &                                  & $p \land ((q \lor r) \lor s) $                     &                                  &                                           & \text{Given}            \\
        2. &                                          &                                  & \cellcolor{OliveGreen!10} $ (p \land (q \lor r)) $ & \cellcolor{OliveGreen!10} $\lor$ & \cellcolor{OliveGreen!10} $ (p \land s) $ & \text{Distributive Law} \\
        3. & \cellcolor{OliveGreen!10}$((p \land q) $ & \cellcolor{OliveGreen!10} $\lor$ & \cellcolor{OliveGreen!10} $(p \land r)) $          & $\lor$                           & $ (p \land s) $                           & \text{Distributive Law} \\
        4. & $((p \land q) $                          & $\lor$                           & \cellcolor{OliveGreen!10}$ (p \land s)) $          & $\lor$                           & \cellcolor{OliveGreen!10} $ (p \land r) $ & \text{Associative Law}  \\
    \end{tabular}
\end{center}

\begin{Note}
    \textbf{Note:} Above is a random example contrived off of expanding a starting expression with laws.
\end{Note}

\begin{Tip}

\end{Tip}


\textbf{Reasoning:}


\noindent
\textbf{Questions:}\\
Of the following, which is DNF and which is CNF, or both?

\begin{enumerate}
    \item $P\land Q\lor \neg W$
\end{enumerate}