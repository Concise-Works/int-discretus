\subsection{Set equivalences}

\noindent
The same way we say, ``it will rain today'' is a proposition, so is,\\
``a set $S$ has an element $x$.'' Using this knowledge we can apply logical equivalences.\\

\noindent
Observe the statement, ``$A=\{1,2\}$ and $B=\{3,4\}$; $A \cap B$.'' Breaking it down:
\begin{itemize}
    \item $A \cap B$ means $x\in A \land y\in B$.
    \item $x\in A$ means $x$ equals ``1 or 2.''
    \item $y\in B$ means $y$ equals ``3 or 4.''
\end{itemize}

\noindent
Describing ``$A \cap B$'' in terms of ``$x\in A \land y\in B$'' allows us to manipulate the expression.\\

\newpage

\noindent
Just like we had equivalence laws for propositions, we have equivalence laws for sets.\\

\noindent
{\Large \textbf{Set Equivalences:}\\}

\noindent
\begin{tabular}{|p{2cm}|l|l|}
    \hline
    \cellcolor{OliveGreen!10} Idempotent:                                            & $A\cup A= A$                                                    & $A\cap A= A$                                 \\
    \hline
    \cellcolor{OliveGreen!10} Associative:                                           & $(A\cup B) \cup C = A \cup (B \cup C)$                          &
    $(A\cap B) \cap C = A \cap (B\cap C)$                                                                                                                                                             \\
    \hline
    \cellcolor{OliveGreen!10} Commutative:                                           & $A\cup B= B \cup A$                                             & $A\cap B = B \cap A$                         \\
    \hline
    \cellcolor{OliveGreen!10} Distributive:                                          & $A\cup (B\cap C) = (A\cup B) \cap (A\cup C)$                    & $A\cap (B\cup C) = (A\cap B) \cup (A\cap C)$ \\
    \hline
    \cellcolor{OliveGreen!10} Identity:                                              & $A\cup \O = A$                                                  & $A\cap U= A$                                 \\
    \hline
    \cellcolor{OliveGreen!10} Domination:                                            & $A\cap \O = \O$                                                 & $A\cup U=A$                                  \\
    \hline
    \cellcolor{OliveGreen!10} Double Negation:                                       & \multicolumn{2}{l|}{$\overline{\overline{A}}=A$}                                                               \\
    \hline
    \cellcolor{OliveGreen!10}\multirow{2}{*}                                         & $A\cap \overline{A} = \O$                                       & $A\cup \overline{A} = U$                     \\
    \cellcolor{OliveGreen!10}\raisebox{.8\normalbaselineskip}[0pt][0pt]{Complement:} & $\overline{U} = \O$                                             & $\overline{\O} = U$                          \\
    \hline
    \cellcolor{OliveGreen!10} De Morgan's Laws:                                      & $\overline{A\cup B} =  \overline{A} \cup \overline{B}$          &
    $\overline{A\cap B} = \overline{A} \cup \overline{B}$
    \\
    \hline
    \cellcolor{OliveGreen!10} Absorption:                                            & $A\cup (A\cap B) = A $                                          & $A\cap (A\cup B) = A$                        \\
    \hline
    \cellcolor{OliveGreen!10} Subset:                                                & \multicolumn{2}{l|}{$A\subseteq B = x\in A \rightarrow x\in B$}                                                \\
    \hline
    \cellcolor{OliveGreen!10} Union:                                                 & \multicolumn{2}{l|}{$A\cap B = x\in A \land x\in B$}                                                           \\
    \hline
    \cellcolor{OliveGreen!10} Intersection:                                          & \multicolumn{2}{l|}{$A\cup B = x\in A \lor x\in B$}                                                            \\
    \hline
\end{tabular}

\vspace{1em}

\noindent
\textbf{For Example:}

\textit{\textbf{Prove:} $A\cup (B\cap C) = (A\cup B) \cap (A\cup C)$.}

\begin{center}
    \begin{tabular}{l l m{.1mm} l m{.1mm} l l}
        1. & $A$ & $\cup$ & $ (B\cap C) $ &  &  & \text{Given} \\
        1. & $A$ & $\cup$ & $ (B\cap C) $ &  &  & \text{Given} \\
    \end{tabular}
\end{center}