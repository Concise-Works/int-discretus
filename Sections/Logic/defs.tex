\subsection{Introduction to Logic}
\vspace{1em}
Observe the claim below:
\vspace{1em}
\begin{center}
    \Large
    \textit{``You're a rotten cook and your breath is sickly."}
\end{center}
\vspace{1em}

\noindent
This \textbf{assertion} is a \textbf{statement} or \textbf{proposition} that is either \underline{\textbf{TRUE} or \textbf{FALSE}}.\\

\vspace{-1em}
\begin{Note}
    \textbf{Note:} claim, assertion, statement, and proposition are all synonyms.
\end{Note}

\vspace{1em}
\noindent
\textbf{Most Importantly:}
\begin{itemize}
    \item Am I actually a rotten cook?
    \item Is my breath really that sickly?
\end{itemize}

\noindent
We can boil down these two propositions to the following:
\begin{itemize}
    \item $R$: You're a rotten cook
    \item $S$: Your breath is sickly
\end{itemize}

\vspace{1em}
\noindent
The original claim can be rewritten as:
\begin{itemize}
    \item ``$R$ and $S$'' formally written, ``$R \land S$'', ``$\land$'' shorthand for ``AND.''\\
          \underline{\textbf{Both} proposition must be true} for the claim to be true.
\end{itemize}

\vspace{1em}
\noindent
Altering the claim to \textit{``You're a rotten cook or your breath is sickly''} yields:
\begin{itemize}
    \item ``$R$ or $S$'' formally written, ``$R \lor S$'', ``$\lor$'' shorthand for ``OR.''\\
          \underline{\textbf{At least one} of the proposition must me true} for the claim to be true.\\\\
          ``\textit{Maybe you're a rotten cook, maybe your breath is sickly, maybe both}.''
\end{itemize}

\begin{Tip}
    To remember the difference between ``$\land$'' and ``$\lor$'', think: $\land$ looks like an ``A''
    without the line for ``AND''.\\

    \noindent
    Also remember ``And'' is CONJUNCTION, ``OR'' is DISJUNCTION. To
    help, think conjunction means ``conjoin'' to join together,\\ ``I'm putting together
    one and one, I'm \textit{conjoining} them.''\\
\end{Tip}

\noindent
Observe the claim below:
\vspace{1em}
\begin{center}
    \Large
    \textit{``I'm not a rotten cook, but I admit my breath is sickly."}
\end{center}
\vspace{1em}

\noindent
The above states, ``Not $R$'' formally written, ``$\neg R$'', ``$\neg$'' shorthand for ``NOT''.
\underline{``but'' in this context is a conjunction.} The claim writes as: ``$\neg R \land S$''.\\

\begin{Tip}
    Try to uncover what a statement is truly saying, not literally.
    In language we often obfuscate sentences to hide intent or to be more polite.\\
\end{Tip}

\noindent
Observe the claim below:
\vspace{1em}
\begin{center}
    \Large
    \textit{``Either I'm a rotten cook or my breath is sickly,\\ not both."}
\end{center}
\vspace{1em}

\noindent
This is an example of the \textbf{exclusive OR} (XOR), we'll use the symbol ``$\oplus$''.\\
Written as ``$R \oplus S$'', \underline{\textbf{only one} of the proposition can be true, not both.}\\

\noindent
This is more obvious in statements such as:
\begin{itemize}
    \item ``They either went to the party or stayed home.''
    \item ``You are either with me or against me.''
    \item ``They either bought lunch or saved their money.''
\end{itemize}

\noindent
We call ``$\land, \lor, \neg, \oplus$'', \textbf{logical operators}.

\begin{Def}[Logical Operators]{def:logical}
    A symbol that represents a logical operation.
    \begin{itemize}
        \item $\land$: AND, both must be true.
        \item $\lor$: OR, at least one must be true.
        \item $\neg$: NOT, negation, opposite.
        \item $\oplus$: XOR, exclusive OR, either or, but not both.
    \end{itemize}
\end{Def}

\newpage

\noindent
\subsection{Statements vs. Predicates:}
Below reads, \underline{\textbf{``4 is less than 2.''} This is false}, but still \textbf{is a statement}:
\begin{center}
    \large
    $4<2$
\end{center}
A statement evaluates to either true or false, \textbf{predicates don't}. Such as:
\begin{center}
    \large
    $x<2$
\end{center}
\textbf{This is a predicate}, it's neither true or false until we assign a value to $x$.\\

\begin{Def}[Proposition]{def:proposition}
    A statement that is either true or false, independent of any variables.
\end{Def}
\begin{Def}[Predicate]{def:predicate}
    A statement where its truth values depend on one or more variables.
\end{Def}

